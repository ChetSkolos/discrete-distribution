% LaTeX document describing the sampling algorithm.
% Use pdflatex to generate a PDF.

\documentclass{article}

\usepackage{amsmath}

\title{Algorithm for Sampling from Discrete Distributions}
\author{D\'avid P\'al}

\begin{document}

\maketitle

\begin{abstract}
This short note describes an algorithm for drawing random samples from any
discrete (categorical) distribution, given as an input, in $O(1)$ time per
sample.  If the distribution has support on $N$ elements, the algorithm needs
$O(N)$ memory and $O(N)$ pre-processing time.
\end{abstract}

\section{Introduction}
\label{section:introduction}

Generating random numbers from various probability distributions is an
important task in many scientific and industrial applications. In this short
note, we describe an algorithm for generating random numbers from a discrete
distributions, also called \emph{categorical distribution}~\cite{wikipedia}.

A discrete distribution is specified by $N$ non-negative real numbers
$p_1, p_2, \dots, p_N$ that satisfy
$$
p_1 + p_2 + \dots + p_N = 1 \; .
$$
A sample from the distribution is a number from the set $\{1,2,\dots,N\}$.
Number $i \in \{1,2,\dots,N\}$ is generated with probability $p_i$.  The number
$N$ is called the number of categories or size of the support.

In practical implementations, the algorithm is given as an input
a list of arbitrary non-negative numbers $w_1, w_2, \dots, w_N$
with positive sum and the goal is to generate samples from the discrete
distribution specified by $p_1, p_2, \dots, p_N$ where
$$
p_i = \frac{w_i}{\sum_{i=1}^N w_i} \; .
$$
Transforming $w_1, w_2, \dots, w_N$ to $p_1, p_2, \dots, p_N$ is easily
achieved in $O(N)$ time as part pre-processing phase.  In the rest of the
paper, we assume this has been done.

The problem of generating random numbers from a distribution is to design an
algorithm that receives numbers $p_1, p_2, \dots, p_N$ as an input,
pre-processes them, and is able to generate any number of independent samples
from the distribution. We make the standard assumption that the algorithm has
access to a random number generator that generates independent random numbers
uniformly from the unit interval $[0,1]$ in $O(1)$ time per sample.

Naive, but fairly common, algorithm requires $O(N)$ pre-processing time, $O(N)$
memory, and can generate sample in $O(\log N)$ time.\footnote{All time and
memory complexities are in the worst-case sense.} As of year 2015, the naive
algorithm is used in many implementations of the C++ Standard
Library~\cite{STL}.  We recap the algorithm in
Section~\ref{section:naive-algorithm}.

In Section~\ref{section:fast-algorithm}, we describe a faster algorithm that
uses $O(N)$ memory, has $O(N)$ pre-processing time, but requires only $O(1)$
time to generate a sample.  The space and time complexities are clearly
optimal.

\section{Naive Algorithm}
\label{section:naive-algorithm}

In the pre-processing phase, the naive algorithm computes prefix sums $s_0,
s_1, s_2, \dots, s_N$ where $s_0 = 0$ and for $i=1,2,\dots,N$
$$
s_i = p_1 + p_2 + \dots + p_i \; .
$$
The prefix sums can be computed in $O(N)$ time and use $O(N)$ memory.

To generate a sample the algorithm makes a single call to the random number
generator and let $X$ be the number it obtains. Using
binary search, the algorithm finds an index $i \in \{1,2,\dots,N\}$ such that
$$
s_{i-1} \le X \le s_i \; .
$$
It is not hard to see that index $i$ is chosen with probability $s_i - s_{i-1}
= p_i$.

\section{Fast Algorithm}
\label{section:fast-algorithm}

The faster algorithm splits the input

TODO

\end{document}
